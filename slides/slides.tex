\documentclass[10pt, xcolor={dvipsnames}, aspectratio = 169]{beamer}
%\usetheme{Berlin}
\usefonttheme{serif}
\usepackage{graphicx}
\usepackage{amsmath}
\usepackage{hyperref}
\usepackage[absolute,overlay]{textpos}
\usepackage{mathrsfs}
%\usepackage{tikz}
%\usetikzlibrary{shapes.geometric, arrows}
\usepackage[font=tiny]{caption}
%\usepackage{mathrsfs}
\usepackage[style=science]{biblatex}
%\usepackage{standalone}

%\bibliography{ref}

\mode<presentation>
{
	\definecolor{nmsured}{RGB}{137,18,22}
	\setbeamercolor{frametitle}{fg=White, bg=nmsured}
	\setbeamercolor{title}{fg=White, bg=nmsured}
	\setbeamercolor{background canvas}{bg=White}
	\setbeamercolor{section number projected}{bg=White, fg=nmsured}
	\setbeamercolor{subsection number projected}{bg=White, fg=nmsured}
	\setbeamertemplate{items}{\color{nmsured}$\blacksquare$}
	\setbeamercolor{text}{fg=nmsured}
	\setbeamertemplate{footline}[frame number]
	\setbeamertemplate{caption}[numbered]
}

\title{NMSU Update}
\author{Dinupa}

\begin{document}

	\begin{frame}
		\maketitle
	\end{frame}

	\begin{frame}[fragile]{Qt App fro Hodoscope Monitoring}
		\begin{textblock}{14.0}(1.0, 2.0)
			\begin{itemize}
				\item \verb|Qt5| framework(\verb|python/c++|) was used.
				\item \verb|Qt| frmaework is;
				\begin{itemize}
					\item Cross-platform software for creating graphical user interfaces (KDE Plasma DE).
					\item Easy-to-read code
					\item Rich choice of modules.
					\item Multiple libraries and high level of control.
					\item API for easier development.
				\end{itemize}
				\item Drawback: can be slow with \verb|ssh|. Better to use \href{https://github.com/Textualize/textual}{textual}.
				\item \href{https://github.com/dinupa1/hodomon}{github}
			\end{itemize}
		\end{textblock}
	\end{frame}

	\begin{frame}[fragile]{CoDaS-HEP: Columnar Data Analysis}
		\begin{textblock}{14.0}(1.0, 2.0)
			\begin{itemize}
				\item A common feature among all array-oriented languages (except \verb|Fortran 90|) is that they are also interactive languages.
				\item Typically, you perform one operation on a whole dataset, see what that does to the distribution, then apply another (easy to debug).
				\item \verb|ROOT| :
				\begin{itemize}
					\item \verb|RDataFrame|
					\item \verb|RVec|
				\end{itemize}
				\item \verb|python|
				\begin{itemize}
					\item \verb|numpy|
					\item \verb|pandas|
					\item \verb|Awkward Array|
				\end{itemize}
				\item \href{https://github.com/jpivarski-talks/2022-08-03-codas-hep-columnar-tutorial#readme}{github}
				\item New kid in the block: \verb|julia| (\url{https://github.com/JuliaLang/julia})
			\end{itemize}
		\end{textblock}
	\end{frame}

	\begin{frame}[fragile]{CoDaS-HEP: Vectorization/ Optimization}
		\begin{textblock}{14.0}(1.0, 2.0)
			\begin{itemize}
				\item Vectorization;
				\begin{figure}
					\centering
					\includegraphics[width=14.0cm]{pic2}
				\end{figure}
			\end{itemize}
		\end{textblock}
	\end{frame}

	\begin{frame}[fragile]
		\begin{textblock}{14.0}(1.0, 2.0)
			\begin{itemize}
				\item Parallel processing: \verb|OpenMP| for multi-threading.
				\item \href{https://github.com/tgmattso/ParProgForPhys}{github}
				\begin{figure}
					\centering
					\includegraphics[width=10.0cm]{pic3}
				\end{figure}
			\end{itemize}
		\end{textblock}
	\end{frame}

	\begin{frame}[fragile]{CoDaS-HEP: Neural Networks}
		\begin{textblock}{14.0}(1.0, 2.0)
			\begin{itemize}
				\item \href{https://github.com/savvy379/codashep_ml_2022}{github}
				\begin{figure}
					\centering
					\includegraphics[width=12.0cm]{pic4}
				\end{figure}
			\end{itemize}
		\end{textblock}
	\end{frame}

	\begin{frame}
		\begin{textblock}{12.0}(1.0, 1.0)
			\begin{figure}
					\centering
					\includegraphics[width=12.0cm]{pic5}
				\end{figure}
		\end{textblock}
	\end{frame}

	\begin{frame}[fragile]{For SpinQuest}
		\begin{textblock}{14.0}(1.0, 2.0)
			\begin{itemize}
				\item Vectorize/Parallelize the \verb|KTracker|.
				\item Use \verb|GNN| for track building.
			\end{itemize}
		\end{textblock}
	\end{frame}

\end{document}